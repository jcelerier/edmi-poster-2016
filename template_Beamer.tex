%%%%%%%%%%%%%%%%%%%%%%%%%%%%%%%%%%%%%%
% Multiplicative domain poster
% Created by Nathaniel Johnston
% August 2009
% http://www.nathanieljohnston.com/2009/08/latex-poster-template/
%%%%%%%%%%%%%%%%%%%%%%%%%%%%%%%%%%%%%%

\documentclass[final]{beamer}
\usepackage{fontspec}
\usepackage[scale=1]{beamerposter}
\usepackage{graphicx}			% allows us to import images
\usepackage[french]{babel}
\usepackage{microtype}
\usepackage{default}


%-----------------------------------------------------------
% Custom commands that I use frequently
%-----------------------------------------------------------

\newcommand{\mainfont}[1]{
	\sffamily{
	#1
	}
	}

%-----------------------------------------------------------
% Define the column width and poster size
% To set effective sepwid, onecolwid and twocolwid values, first choose how many columns you want and how much separation you want between columns
% The separation I chose is 0.024 and I want 4 columns
% Then set onecolwid to be (1-(4+1)*0.024)/4 = 0.22
% Set twocolwid to be 2*onecolwid + sepwid = 0.464
%-----------------------------------------------------------

\newlength{\sepwid}
\newlength{\onecolwid}
\newlength{\twocolwid}
\setlength{\paperwidth}{48in}
\setlength{\paperheight}{36in}
\setlength{\sepwid}{0.024\paperwidth}
\setlength{\onecolwid}{0.22\paperwidth}
\setlength{\twocolwid}{0.464\paperwidth}
\setlength{\topmargin}{-0.5in}
\usetheme{confposter}
\usepackage{exscale}

%-----------------------------------------------------------
% The next part fixes a problem with figure numbering. Thanks Nishan!
% When including a figure in your poster, be sure that the commands are typed in the following order:
% \begin{figure}
% \includegraphics[...]{...}
% \caption{...}
% \end{figure}
% That is, put the \caption after the \includegraphics
%-----------------------------------------------------------

\usecaptiontemplate{
\small
\structure{\insertcaptionname~\insertcaptionnumber:}
\insertcaption}

%-----------------------------------------------------------
% Define colours (see beamerthemeconfposter.sty to change these colour definitions)
%-----------------------------------------------------------

\setbeamercolor{block title}{fg=ngreen,bg=white}
\setbeamercolor{block body}{fg=black,bg=white}
\setbeamercolor{block alerted title}{fg=white,bg=dblue!70}
\setbeamercolor{block alerted body}{fg=black,bg=dblue!10}
%\setmainfont{Fira Sans}
%\setsansfont{Fira Sans}
%\setsansfont{Fira Code}
%-----------------------------------------------------------
% Name and authors of poster/paper/research
%-----------------------------------------------------------

\title{Techniques vidéo-ludiques pour la conception d'un logiciel auteur multimédia}
\author{Jean-Michaël Celerier}
\institute{Laboratoire Bordelais de Recherche en Informatique, Blue Yeti}

%-----------------------------------------------------------
% Start the poster itself
%-----------------------------------------------------------
% The \rmfamily command is used frequently throughout the poster to force a serif font to be used for the body text
% Serif font is better for small text, sans-serif font is better for headers (for readability reasons)
%-----------------------------------------------------------

\begin{document}
\begin{frame}[t]
  \begin{columns}[t]												% the [t] option aligns the column's content at the top
    \begin{column}{\sepwid}\end{column}			% empty spacer column
    \begin{column}{\onecolwid}
      \begin{alertblock}{Problématique}
        \mainfont{}
      \end{alertblock}
      \vskip2ex
      \begin{block}{Partitions interactives}
        \mainfont{État de l'art, domaine du problème}
      \end{block}
      \vskip2ex
      \begin{block}{Modèles pour logiciels auteurs}
        \mainfont{}
        % penser à citer bouquin que j'avais montré à simon sur CSP
        MVP, MVVM, ECS, voir aussi modèle de Michel Baudouin-Lafont sur approche document interactive.
      \end{block}
    \end{column}

    \begin{column}{\sepwid}\end{column}
    \begin{column}{\twocolwid}
      \begin{columns}[t,totalwidth=\twocolwid]
        \begin{column}{\onecolwid}\vspace{-.69in}
          \begin{block}{Entités}
            \mainfont{Identification (avec cache), chemins complets type-safe (avec cache), hiérarchie, document}
          \end{block}
          \begin{block}{ECS hiérarchique}
          	\mainfont{Création automatique, gestion des ressources, sauvegarde, définition de l'E, du C, du S dans notre cas}
          \end{block}
        \end{column}
        \begin{column}{\onecolwid}\vspace{-.69in}
        	\begin{block}{Génération de boilerplate avec CMake}
        		\mainfont{Intégration au système de plug-ins, merging des composants, factories en temps linéaire et allouées statiquement pour build statique vs quadratique m fois n pour build dynamique}
        	\end{block}
        	\begin{block}{Édition en temps réel avec rollback}
        		\mainfont{}
        	\end{block}
        \end{column}
      \end{columns}
      \vskip2.5ex
      \begin{alertblock}{i-score}
      	
      \end{alertblock}
  \end{column}
  \begin{column}{\sepwid}\end{column}			% empty spacer column
  \begin{column}{\onecolwid}
    \begin{block}{Résultats}
      \mainfont{}
    \end{block}
    \vskip2ex
    \begin{block}{Informations complémentaires}
    	% Site web
    	% Article
      \small{\mainfont{Articles:
      \begin{itemize}
        \item -ossia
        \item -iscore
      \end{itemize}
      \vspace{0.1in}\noindent i-score peut être téléchargé sur
      \begin{itemize}
        \item www.i-score.org
      \end{itemize}}}
    \end{block}
    \vskip2ex
    \begin{block}{References}
    \end{block}
    \vskip2ex
  \end{column}
  \begin{column}{\sepwid}\end{column}			% empty spacer column
 \end{columns}
\end{frame}
\end{document}

