%%%%%%%%%%%%%%%%%%%%%%%%%%%%%%%%%%%%%%
% This poster uses a theme taken from Nathaniel Johnston
% http://www.nathanieljohnston.com/2009/08/latex-poster-template/
%%%%%%%%%%%%%%%%%%%%%%%%%%%%%%%%%%%%%%

\documentclass[final]{beamer}
\usepackage[scale=1.42]{beamerposter}
\usepackage{graphicx}
\usepackage[french]{babel}
\usepackage{microtype}
\usepackage{default}


%-----------------------------------------------------------
% Define the column width and poster size
% To set effective sepwid, onecolwid and twocolwid values, first choose how many columns you want and how much separation you want between columns
% The separation I chose is 0.024 and I want 4 columns
% Then set onecolwid to be (1-(4+1)*0.024)/4 = 0.22
% Set twocolwid to be 2*onecolwid + sepwid = 0.464
%-----------------------------------------------------------

\newlength{\sepwid}
\newlength{\onecolwid}
\newlength{\twocolwid}
\setlength{\paperwidth}{48in}
\setlength{\paperheight}{36in}
\setlength{\sepwid}{0.024\paperwidth}
\setlength{\onecolwid}{0.22\paperwidth}
\setlength{\twocolwid}{0.464\paperwidth}
\setlength{\topmargin}{-0.5in}
\usetheme{confposter}
\usepackage{exscale}

%-----------------------------------------------------------
% The next part fixes a problem with figure numbering. Thanks Nishan!
% When including a figure in your poster, be sure that the commands are typed in the following order:
% \begin{figure}
% \includegraphics[...]{...}
% \caption{...}
% \end{figure}
% That is, put the \caption after the \includegraphics
%-----------------------------------------------------------

\usecaptiontemplate{
\small
\structure{\insertcaptionname~\insertcaptionnumber:}
\insertcaption}

%-----------------------------------------------------------
% Define colours (see beamerthemeconfposter.sty to change these colour definitions)
%-----------------------------------------------------------

\setbeamercolor{block title}{fg=DarkGray,bg=white}
\setbeamercolor{block body}{fg=DarkGray,bg=white}
\setbeamercolor{block alerted title}{fg=DarkGray,bg=LightGray}
\setbeamercolor{block alerted body}{fg=DarkGray,bg=LightGray}
%-----------------------------------------------------------
% Name and authors of poster/paper/research
%-----------------------------------------------------------

\title{Techniques vidéo-ludiques pour la conception d'un logiciel auteur multimédia}
\author{Jean-Michaël Celerier}
\institute{Laboratoire Bordelais de Recherche en Informatique, Blue Yeti}

\begin{document}
\begin{frame}[t]
   \begin{columns}[t]
     \begin{column}{\sepwid}\end{column}
     \begin{column}{\onecolwid}
      \begin{block}{Problématique}
          \begin{columns}[t]
              \begin{column}{\onecolwid}\justify
                  Comment concevoir un logiciel amené à être utilisé par des artistes tout en servant de plate-forme de recherche pour des technologies multimédia?
                \end{column}
            \end{columns}        
      \end{block}
     \end{column}
     \begin{column}{\sepwid}\end{column}
     \begin{column}{\twocolwid}
         \begin{block}{Méthode}             
             \begin{columns}[t]	                 
                 \begin{column}{\onecolwid}\justify
                     Lorem ipsum dolor sit amet, consectetur adipiscing elit, sed do eiusmod tempor incididunt ut labore et dolore magna aliqua. Ut enim ad minim veniam, 
                     \end{column}
                     \begin{column}{\onecolwid}\justify
                         Duis aute irure dolor in reprehenderit in voluptate velit esse cillum dolore eu fugiat nulla pariatur. Excepteur sint occaecat cupidatat non proident, 
                        \end{column}
                \end{columns}                 
            \end{block}
      \end{column}
      \begin{column}{\sepwid}\end{column}
      \begin{column}{\onecolwid}
          \begin{block}{Résultats}
              posay
            \end{block}
        \end{column}         
   \begin{column}{\sepwid}\end{column}		
  \end{columns}
  \begin{columns}[t]	
      											% the [t] option aligns the column's content at the top
    \begin{column}{\sepwid}\end{column}			% empty spacer column
    \begin{column}{\onecolwid}
      \vskip2ex
       \begin{block}{Partitions interactives}
\begin{figure}
\includegraphics[width=\columnwidth]{images/score.eps}
\caption{Syntaxe d'une partition interactive}
\end{figure}
Possibilités d'écritures forment un langage de programmation structuré axé sur l'organisation temporelle. Possèe la notion de boucles et de hiérarchie, 
mais pas de calcul.
Applications : musique interactive, scénographie et spectacle vivant, contrôle de robots.
Autres approches graphiques : Antescofo, INscore, OpenMusic; ainsi qu'approches programmatiques : Abjad, Tuiles réactives, \dots
\end{block}
      \vskip2ex
      \begin{block}{Modèles pour logiciels auteurs}
Standard : modèle-vue-contrôleur, modèle-vue-présenteur, document-présentation-instrument, modèle-vue-modèle de vue, présentation-abstraction-contrôle, programmation fonctionnelle réactive.
Donnent des responsabilités à différents éléments de l'application et spécifient la communication entre ces éléments et la manière dont une interaction utilisateur affecte le modèle de données.
Principalement orienté pour l'affichage, mais s'adaptent peu à d'autres moadlités.

Problématique de l'édition; c.f. Object-Oriented Programming for Graphics

\end{block}
    \end{column}

    \begin{column}{\sepwid}\end{column}
    \begin{column}{\twocolwid}
      \begin{columns}[t,totalwidth=\twocolwid]
        \begin{column}{\onecolwid}\vspace{-.69in}
            \vskip2ex
            \begin{block}{Entités}
Identification (avec cache), chemins complets type-safe (avec cache), hiérarchie, document
\end{block}
            \vskip2ex
            
          \begin{block}{ECS hiérarchique}
          	Création automatique, gestion des ressources, sauvegarde, définition de l'E, du C, du S dans notre cas
          \end{block}
        \end{column}
        \begin{column}{\onecolwid}\vspace{-.69in}
            \vskip2ex
            \input{cmake}
            \vskip2ex
            
        	\begin{block}{Édition en temps réel avec rollback}
                
        	\end{block}
        \end{column}
      \end{columns}
      \vskip2.5ex
      \begin{alertblock}{i-score}
      	
      \end{alertblock}
  \end{column}
  \begin{column}{\sepwid}\end{column}			% empty spacer column
  \begin{column}{\onecolwid}
    \begin{block}{Résultats}
    \end{block}
    \vskip2ex
    \begin{block}{Informations complémentaires}
    	% Site web
    	% Article
      \small{Articles:
      \begin{itemize}
        \item -ossia
        \item -iscore
      \end{itemize}
      \vspace{0.1in}\noindent i-score peut être téléchargé sur
      \begin{itemize}
        \item www.i-score.org
      \end{itemize}}
    \end{block}
    \vskip2ex
    \begin{block}{Références}
    \end{block}
    \vskip2ex
  \end{column}
  \begin{column}{\sepwid}\end{column}			% empty spacer column
 \end{columns}
\end{frame}
\end{document}

