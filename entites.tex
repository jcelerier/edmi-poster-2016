\begin{block}{Implémentation}
Système d'entités adapté pour \textbf{hiérarchie fixée} dans le modèle : tout ne se compose pas avec tout.
Identification unique fortement typée avec cache~: 
\begin{itemize}
    \item Dans un document par chemins : nécessaire pour gestion undo - redo et identification sur réseau.
    Pattern Commande réparti pour \textbf{édition multi-utilisateurs}.
    \item À un niveau de hiérarchie donnée : performance pour itération.
   \end{itemize}

Création de familles de composants fortement ou faiblement typés selon le besoin, et associés à un élément de modèle : fig.~\ref{fig.uml}.

Contrairement à moteurs de jeu, \textbf{pas de synchro des tick rate} car tous les systèmes sont séparés ; certains systèmes peuvent fonctionner sur le même thread ou sur des threads différents. 
Les systèmes peuvent communiquer entre eux.
\end{block}